\section{Non-Interactive Proofs of Proofs of Work}
Seeing that proofs are linear in the chain length, it is natural to consider if we can do better. NIPoPoW \cite{nipopows} provides the first family of succinct proofs which are logarithmic in the size of the chain and proven secure in the Backbone \cite{backbone} protocol.

There have been previous attempts to create proofs smaller in size than SPV proofs~\cite{KLS}, where a scheme for logarithmic proofs was proposed.  This scheme was later proven insecure~\cite{nipopows}.

\subsection{Terminology}
NIPoPoWs are categorized in two kinds of proofs:

\begin{itemize}
  \item We have a valid chain where the last $k$ blocks (also called the \textit{unstable} part of the chain) are the ones we're claiming. This is called a \textbf{suffix proof}.
  \item We have a valid chain where a specific given block is included in its \textit{stable} part (excluding the last $k$ blocks). This is called an \textbf{infix proof}.
\end{itemize}

\subsection{Assumptions}
An assumption NIPoPoWs make is that the difficulty is constant. This is not true for Bitcoin or Bitcoin Cash.

% further analysis in |
%                     \/
% cite variable difficulty backbone TODO

NIPoPoWs also assume each block contains an interlink data structure, which we'll study shortly. Interlinks too don't exist in Bitcoin or Bitcoin Cash.

In the next section we'll look at how we sidestep all those issues.

\subsection{Levels}
At the heart of the primitive lies the separation of blocks into levels. The level of a block is defined as $\textit{level}(B) = \left \lfloor \log(T) - \log(\sf{id}(B)) \right \rfloor$, where $T$ is the constant difficulty of the blockchain. The genesis block is an exception to this rule as $\textit{level}(Gen) = \infty$. We call a block of level $\mu$ a $\mu$-superblock.

Intuitively, the level of a block is the number of leading zeros of the binary representation of the block id when left padded to the length of $T$. An example of this can be seen on Table~\ref{table:level-counting}.

\begin{table}
  \centering
  \begin{tabular}{|c|c|}
    \hline
    $T$ & 1110000 \\
    \hline
    $\sf{id}(B)$ & \underline{000}1000 \\
    \hline
  \end{tabular}
  \caption{Calculating the level of a block by counting the leading zeros (3 in this case).}
  \label{table:level-counting}
\end{table}

Figure~\ref{fig:hierarchy} shows an example the implied blockchain created from the superblocks.

\begin{figure}
  \centering
  \includegraphics[width=0.9\columnwidth,keepaspectratio]{figures/hierarchical-ledger.png}
  \caption{The hierarchical blockchain.  Higher levels have achieved a lower target (higher difficulty) during mining. All blocks are connected to the genesis block $G$. Source:~\cite{nipopows}}
  \label{fig:hierarchy}
\end{figure}

\subsection{Notation}
The NIPoPoWs paper introduces some notation for talking about blockchains with levels which we'll be using extensively. The notation is widely influenced by Python. Specifically:

\begin{itemize}
  \item $\chain$ denotes a blockchain, with $\chain[0]$ being the genesis block, $\chain[k]$ being the $k$-th first block and $\chain[-k]$ being the $k$-th last block.
  \item $\chain[k:]$ denotes the sub-blockchain starting from the $k$-th block, $\chain[-k:]$ denotes the sub-blockchain starting from the $k$-th last block.
  \item $\chain[:k]$ denotes the sub-blockchain ending before the $k$-th block, $\chain[:-k]$ denotes the sub-blockchain ending before the $k$-th last block.
  \item $\chain[i:j]$ denotes the sub-blockchain starting from the $i$-th block and ending at the $j$-th block. $i$ and $j$ can also be negative numbers similar to above.
  \item $\chain\{B:\}$ denotes the sub-blockchain starting from the block with block id $B$.
  \item $\chain\upchain^\mu$ denotes the sub-blockchain of $\chain$ where all blocks are of level $\mu$ or higher.
\end{itemize}

\subsection{Interlink}
Instead of keeping only the hash of the previous block inside the block header, for every superblock level we keep a pointer to the most recent superblock of that level. The structure containing these pointers is called the interlink. Bitcoin does not support such a structure in the block header but we will study how to sidestep this issue by velvet forking in a few sections.

\begin{table}
  \centering
  \begin{tabular}{|c|c|}
    \hline
    Level & Block \\
    \hline
    $0$ & $C[-2]$ \\
    $1$ & $C[-2]$ \\
    $2$ & $C[-4]$ \\
    $3$ & $C[-8]$ \\
    $\infty$ & $C[0]$ \\
    \hline
  \end{tabular}
  \caption{Interlink of $C[-1]$ from Figure~\ref{fig:hierarchy}}
  \label{table:interlink-example}
\end{table}

It's important to note that the interlink can be encoded as a series of block ids, starting from $0$ up to $\infty$. It can also be compressed by using this series as the leafs of a Merkle tree and taking the Merkle tree root.

Suppose we have a block $B'$ with an interlink stored as $B'.{\sf interlink}$. In order to produce the interlink for a block after $B'$ we make sure to change all pointers from level $0$ up to $level(B')$ to point to $B'$, as $B'$ will be the most recent block at these levels (remember that a block of level $\mu$ is also of level $\mu-1$). We call this procedure {\sf updateInterlink}, which can be seen in detail on Algorithm~\ref{alg.nipopow-interlink}.

\input{algorithms/alg.nipopow-interlink.tex}

\subsection{Suffix Proofs}
Suffix proofs are parameterized by $k$ and $m$. $k$ refers to the number of blocks that need to bury a block for it to be considered stable.

A suffix proof of a chain $\chain$ is constituted of two chains, $\pi$ and $\chi$. The final proof is the concatenation of those two chains $\pi \chi$. $\chi$ always refers to the chain of unstable blocks and is evaluated as $\chi = \chain[-k:]$.

The process for constructing $\pi$ is a little more convoluted. First we have to find the first level $\mu$ where $|\chain\upchain^\mu| \ge m$. We call this level $max\mu$. For this level we take all its blocks except for the last $m$: $\pi_{max\mu} = \chain\upchain^\mu[:-m]$. Then for every level $\mu$ from $max\mu - 1$ to $0$, we take blocks $\pi_\mu = \chain\upchain^\mu [:-m]\{\chain\upchain^{\mu+1}[-m]:\}$.

$\pi$ is then evaluated as the concatenation of all those chains starting from the oldest block:

$$ \pi = \pi_{max\mu} || \pi_{max\mu-1} || \ldots || \pi_0 $$

\begin{figure}
  \centering
  \includegraphics[width=0.9\columnwidth,keepaspectratio]{figures/non-interactive-popow.pdf}
  \caption{Construction of $\pi$ of a suffix proof. $m=3$  Source:~\cite{nipopows}}
  \label{fig:suffix-proof}
\end{figure}

It's important to notice that the chain provided to the verifier is an actual chain: one can start at the end and traverse it until the genesis block by utilizing the interlink of each block, similar to how they would do that on a conventional blockchain by using each block's $\sf previd$.

\subsection{Infix Proofs}
\label{ssec:infix}
For the verifier to be able to determine a predicate on one or more blocks ($\chain' \subseteq \chain$) of our chain, we have to make sure we include them in a proof. A suffix proof is not guaranteed to include all blocks of interest in $\chain'$. In order to include these we have to make sure they are linked to the proof, e.g. that the proof chain is a traversable. To this end, let's assume some arbitrary block $B \in \chain'$. Let's also assume an existing suffix proof $\pi\chi$. We find blocks $E'$ and $E$ on the suffix proof, such that:

\begin{itemize}
  \item $E$ is the next block after $E'$ on the proof
  \item $B$ comes before $E$ on $\chain$
  \item $B$ comes after $E'$ on $\chain$
\end{itemize}

An example of such a triplet of blocks satisfying those conditions can be seen on Figure~\ref{fig:infix-proof}.

\begin{figure}
  \centering
  \includegraphics[width=0.9\columnwidth,keepaspectratio]{figures/infix.pdf}
  \caption{Construction of an infix proof.  Source:~\cite{nipopows}}
  \label{fig:infix-proof}
\end{figure}

We then perform a procedure called $\sf followDown$ in order to figure out which blocks need to be added to the proof in order to link $E$ to $B$. $\sf followDown$ includes blocks on intermediate levels until $B$ is reached. The full algorithm can be seen on Algorithm~\ref{alg.nipopow-infix-follow}.

\input{algorithms/alg.nipopow-infix-follow.tex}

Augmenting the original suffix proof with the new blocks provided by $\sf followDown$ on all our blocks of interest $B \in \chain'$ gives us our final infix proof. Note that as was the case with our suffix proofs, the infix proof is traversable.

% TODO: I never talk about the infix predicate

\subsection{Proof Validation}
It is very easy to validate a proof, similarly to how we validated SPV proofs in the previous chapter: one can just check that the proof is traversable up to a known genesis, and $|\chi| = k$. Validation in the context of NIPoPoWs refers to a single-prover environment.

\subsection{Suffix Proof Verification}
In a multi-prover environment, we noted the need for a way to compare proofs. For SPV proofs this was easy, the length of each chain could be compared (as long as both chains were valid). For NIPoPoWs however the process of comparing is not as straightforward, if we consider that the number of blocks is not directly connected to chain length.

In Algorithm \ref{alg.nipopow-maxchain} we show how two proofs can be compared. The algorithm finds the Lowest Common Ancestor $b$ from the stable part of both proofs, and then calculates \textsf{best-arg} on each proof from $b$ onwards. The proof with the largest \textsf{best-arg} is decided to be the better proof.

\input{algorithms/alg.nipopow-maxchain.tex}

Knowing how to compare two proofs, the only thing a verifier has to do is actually compare all the proofs it receives (which are collected in the set $\mathcal{P}$) and only keep the best one. This is shown in Algorithm \ref{alg.nipopow-verifier}. As an extra step, after picking the best proof $(\tilde\pi, \tilde\chi)$, this verifier for suffix proofs evaluates a predicate $\tilde{Q}$ on the suffix, or unstable part of the best proof it received.
\input{algorithms/alg.verifier-lite.tex}

\subsection{Velvet Forks}
Velvet forks~\cite{nipopows,velvet} describe a formalization of adding arbitrary data inside blocks in order to allow potential applications without sacrificing the backwards compatibility of the blockchain.

Miners who are willing to contribute to the fork can add data of interest in the form of coinbase transaction data. 

Backwards compatibility is achieved by not changing the consensus rules, meaning that set of acceptable blocks does not change. So any block that was acceptable remains acceptable even if it does not contain any data concerning the fork, or if it contains invalid data.

\subsection{User-Activated Velvet Forks}
In case miners aren't interested in including such data, users can also create such a fork by making a kind of transaction called velvet transaction. In a velvet transaction a user includes any data of interest in unspendable transaction outputs (like OP\_RETURN).

For the consumer of such data, the only difference is that they have to look inside the whole block to find it, not only inside the coinbase data.

Such forks come at the cost of making such transactions, because the user who makes the fork needs to pay transaction fees every time they wish to add data to the blockchain.

\subsection{Velvet NIPoPoWs}
Since anyone can post such transactions on the blockchain, we have to make sure that the commitment is actually true before we can use it. In order to do that we maintain our own version of the interlink for each block which we know is correct called $\sf realLink$. Then for every block, we compare its commitments (there may be many) with the Merkle Tree root of our $\sf realLink$. If there is a valid commitment we say that the block has a valid interlink. We store the full interlink as $\sf realLink[id(block)]$.

We already know how it's essential that our proof forms a blockchain that can be traversed from start to end. In order to make our proof traversable, whenever we include a block we have to make sure it connects validly to the previous one either by (a) using the regular $\sf previd$ inside the block or (b) using a valid interlink. If we use the $\sf previd$ to link back to the previous block then all the information someone needs to verify the traversability is already there and we don't need to add anything extra. In the case we use the interlink however, we need to provide the Merkle Tree proofs for:

\begin{itemize}
  \item The transaction containing the valid interlink commitment.
  \item The interlink level we use for the connection.
\end{itemize}

If our chain is not traversable the proof is automatically invalid.

We'll now look at the concrete implementation of the prover.

\subsection{Invalid Interlink Handling}
The original NIPoPoWs paper~\cite{nipopows} gives us some insight into how to handle blocks with invalid interlinks. Let's look at where we need to make changes starting with suffix proofs.

For suffix proofs we need a way to obtain the upchain of a chain, denoted as $\chain\upchain^\mu$. We will now define a procedure to programatically obtain the upchain of a chain called $\sf find \chain\upchain^\mu$.

\subsubsection{followUp}
$\sf followUp$ takes a block $b$ and a level $\mu$ as parameters. Starting from $b$ it traverses the chain until it reaches another block of level $\mu$ called $B$. In doing so, it is only allowed to use valid pointers. It will only follow a pointer from a block's interlink at level $\mu$ if there is a valid interlink in that block. Otherwise, the only option is to follow the previous block pointer ($\sf previd$). Once $B$ is reached, it is returned alongside the blocks that were traversed as a blockset called $\sf aux$.

\input{algorithms/alg.nipopow-velvet-follow.tex}

%TODO figure with straight and dropdown cases

\subsubsection{find$\chain\upchain^\mu$}
Now that we have a way to go back on a level, we can utilize it to construct the entire traversable level $\mu$ up to block $b$, starting from the end of the chain $\chain[-1]$: this is what ${\sf find \chain\upchain^\mu}(b)$ accomplishes. It works by repeatedly calling $\sf followUp$ on the oldest block it has and including the result in its final chain.

\input{algorithms/alg.nipopow-velvet-upchain.tex}

\subsubsection{Velvet Infix Proofs}

The infix prover also changes to account for invalid or nonexistent interlinks. On \textsf{followDown} we have to make sure that any interlinks we use are valid. Thus, for any block with an invalid interlink we are forced to use its \textsf{previd} instead to find its previous block, in hopes that the previous block will contain a usable interlink.

Another problem is that our block of interest may have an invalid interlink, thus it is not guaranteed that \textsf{followDown} will be enough. To mitigate this, we make sure to create a valid path of valid interlinks back to our suffix proof block, with a procedure called \textsf{goBack}, seen in Algorithm \ref{alg.nipopow-velvet-infix-go-back}, which is very works similarly to \textsf{followDown}, but instead of attempting to downgrade to lower levels it attempts to upgrade to higher levels.

\input{algorithms/alg.nipopow-velvet-infix-go-back.tex}

\chapter{Introduction}

\section{Motivation}
\emph{Cryptocurrencies} are digital assets that utilize cryptography in order to allow value transfer, without the need of a central party or trusted authority. The technology originally appeared in 2008 in a paper by Satoshi Nakamoto~\cite{bitcoin} as Bitcoin, along with a reference implementation in C++. It didn't take long until a community of enthusiasts and cryptographers embraced the technology and started studying it and using it extensively. New cryptocurrencies based on Bitcoin's ideas and codebase started popping up, among them most notably Litecoin and Dogecoin.

This work clearly was a huge inspiration. In 2014 Ethereum~\cite{ethereum} appeared, which aimed to do much more than just value transfers: it built on Nakamoto's ideas in order to build a world computer. Programs called \emph{smart contracts} could be stored and run on a decentralized manner. Such smart contracts gave us the ability to write immutable contracts, where \emph{code is law}.

Few years later and the landscape is completely changed. More and more cryptocurrencies are created every single day. Public interest and prices have skyrocketed. There is lots of optimism about the future decentralized techologies such as Bitcoin can bring, mainly a democratization of money, usually called "banking the unbanked".

However, being more widespread and popular surfaced some problems the most important of which is scalability. Technically, each cryptocurrency has a \emph{blockchain}, which is literally a chain of \emph{blocks} linked together like a linked list. These blocks contain \emph{transactions}. Every transaction needs to be recorded and stored in a block, and everyone has to know about it. As a result, the Bitcoin blockchain is 185GB at the time of writing. The Ethereum blockchain comes at 667GB. Typically, for someone to participate on the network, and do actions like send transactions they have to download the whole blockchain. However at such rates it is very time-consuming and resource-intensive or even impossible for someone to download a chain. So called \emph{lite nodes} that don't need to download the whole chain do exist, but at best they need information linear to the size of the chain, so they're a constant-factor improvement.

New cryptocurrencies with interesting features pop up all the time. There's long been an interest in implementing sidechains~\cite{sidechains}, as a way to interoperate between two blockchains. One should be able to trustlessly transfer his Bitcoin to another chain and use it there, and transfer it back if he so desires. TODO

\section{Related Work}
\section{Objectives}
\section{Structure}
